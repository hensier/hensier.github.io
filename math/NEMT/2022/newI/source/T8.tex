\documentclass[UTF8]{ctexart}
\usepackage{amsmath}
\begin{document}
\thispagestyle{empty}
\pagestyle{empty}

\noindent 已知正四棱锥的侧棱长为 $l$,其各顶点都再同一球面上,若该球的体积为 $36 \pi$,且 $3 \le l \le 3 \sqrt 3$,则该正四棱锥体积的取值范围是

\noindent A. $[18,\frac{81}{4}]$ \,
B. $[\frac{27}{4},\frac{81}{4}]$ \,
C. $[\frac{27}{4},\frac{64}{3}]$ \,
D. $[18,27]$ \\

\noindent 解:设四棱锥底面边长为 $\sqrt 2a$,则

$$(3+\sqrt{9-a^2})^2+a^2=l^2$$

\noindent 化简得

$$a^2=\frac{36l^2-l^4}{36}$$

\noindent 设 $t=l^2$,棱锥体积为 $V(t)$,则

\begin{align*}
V(t) &= 2a^2+\frac{2a^2}{3} \times \sqrt{9-a^2} \\
&= \frac{t^2}{324}(36-t) \qquad  \qquad (t \in [9,27])
\end{align*}

\noindent 对上式求导得

\begin{align*}
V'(t) &= \frac{1}{324} \times [2t \times (36-t)-t^2] \\ 
&= \frac{24t-t^2}{108}
\end{align*}

\noindent 设能让 $V(t)$ 取极值的自变量为 $t_0$,则

$$24t_0-t_0^2=0 \Rightarrow t_0=0 \text{(舍)或 } t_0=24$$

\noindent 由此可知 $V(t)$ 在 $(9,24)$ 上为增函数,在 $(24,27)$ 上为减函数.

\noindent 因而

$$V_{\max}=V(24)=\frac{24^2}{324}(36-24)=\frac{64}{3}$$

$$V_{\min}=\min\{V(9),V(27)\}=\min\{\frac{9^2}{324}(36-9),\frac{27^2}{324}(36-27)\}=\frac{27}{4}$$

\noindent 故选 C.

\end{document}