\documentclass[UTF8]{ctexart}
\usepackage{amsmath}
\newcommand{\lt}{<}
\newcommand{\gt}{>}
\begin{document}
\thispagestyle{empty}
\pagestyle{empty}

\noindent 记 $\triangle ABC$ 的内角 $A,B,C$ 的对边分别为 $a,b,c$. 已知 $\dfrac{\cos A}{1+\sin A}=\dfrac{\sin 2B}{1+\cos 2B}$.

\noindent (1) 若 $C=\dfrac{2 \pi}{3}$,求 $B$. \\

\noindent (2) 求 $\dfrac{a^2+b^2}{c^2}$ 的最小值. \\

\noindent 解:化简题中条件

\begin{align*}
\dfrac{\cos A}{1+\sin A} &= \dfrac{\sin 2B}{1+\cos 2B} \\
2 \sin B \cos B(1+ \sin A) &= 2 \cos A \cos ^2 B
\end{align*}

\noindent 这里需讨论 $\cos B$ 是否为 $0$. 当 $\cos B=0$ 即 $B=\dfrac{\pi}{2}$ 时,上述等式成立.

\noindent 当 $\cos B \neq 0$ 时

\begin{align*}
2 \sin B(1+ \sin A) &= 2 \cos A \cos B \\
2(\cos A \cos B-\sin A \sin B) &= 2 \sin B \\
2 \cos(A+B) &= 2 \sin B \\
\sin B+\cos C=0
\end{align*}

\noindent 综上,$B=\dfrac{\pi}{2}$ 或 $\sin B+\cos C=0$. \\

\noindent (1) 因为 $C=\dfrac{2 \pi}{3} \gt \dfrac{\pi}{2}$,所以 $B \lt \dfrac{\pi}{2}$,因此只能有 $\sin B=-\cos C=\cos \dfrac{\pi}{3}=\dfrac{1}{2}$. \\

\noindent 故 $B=\dfrac{\pi}{6}$. \\

\noindent (2) 分类讨论:

\noindent ① 当 $B=\dfrac{\pi}{2}$ 时,原式 $=\dfrac{2a^2+c^2}{c^2} \gt 1$. \\

\noindent ② 当 $B \lt \dfrac{\pi}{2}$ 时:

\begin{align*}
\sin A &= \sin(B+C)=\sin B \cos C+\sin C \cos B \\
&= -\sin^2 B+\sqrt{(1-\cos^2 C)(1-\sin^2 B)} \\
&= 1-2 \sin^2 B \\
&= 1-2 \cos^2 C \\
\end{align*}

\noindent 因而

\begin{align*}
\dfrac{a^2+b^2}{c^2} &= \dfrac{(1-2 \cos^2 C)^2+\cos^2 C}{\sin^2 C} \\
&= \dfrac{1-4 \cos^2 C+4 \cos^4 C+\cos^2 C}{\sin^2 C} \\
&= -4 \cos^2 C-1+\dfrac{2}{1-\cos ^2 C} \\
&= 4-4 \cos^2 C+\dfrac{2}{1-\cos^2 C} -5 \\
&= 4 \sin^2 C+\dfrac{2}{\sin^2 C}-5 \\
&\ge 2 \sqrt{4 \times 2}-5 \\
&= 4 \sqrt{2}-5
\end{align*}

\noindent 当且仅当 $\sin^4 C=\dfrac{1}{2}$ 时,上式成立. \\

\noindent 则当 $B \lt \dfrac{\pi}{2}$ 时,$\dfrac{a^2+b^2}{c^2}$ 的最小值为 $4 \sqrt 2-5$.

\noindent 因为 $4 \sqrt 2-5 \lt 1$,所以 $\dfrac{a^2+b^2}{c^2}$ 的最小值为 $4 \sqrt 2-5$.

\end{document}