\documentclass[UTF8]{ctexart}
\usepackage{amsmath}
\newcommand{\lt}{<}
\newcommand{\gt}{>}
\begin{document}
\thispagestyle{empty}
\pagestyle{empty}

\noindent 设 $a=0.1e^{0.1}$,$b=\dfrac{1}{9}$,$c=-\ln 0.9$,则

\noindent A. $a \lt b \lt c$ \,
B. $c \lt b \lt a$ \,
C. $c \lt a \lt b$ \,
D. $a \lt c \lt b$ \\

\noindent 解:本题使用泰勒公式更加方便:

$$f(x)=\sum_{i=0}^{+\infty} \dfrac{f^{(i)}(x_0)}{i!} (x-x_0)^i$$

\noindent 对于函数 $a(x)=e^x$,可在 $x_0=0$ 处进行展开:

\begin{align*}
a(x) &= \sum_{i=0}^{+\infty} \dfrac{a^{(i)}(0)}{i!} x^i \\
&= 1+x+\dfrac{x^2}{2}+\dfrac{x^3}{6}+
\dfrac{x^4}{24}+\cdots
\end{align*}

\noindent 不妨计算到 $2$ 次方项:

$$a(0.1) \approx 1+10^{-1}+5 \times 10^{-3} \approx 1.105$$

\noindent 因此 $a \approx 0.1105$.

\noindent 对于 $b$,显然有 $b=0.\dot 1$.

\noindent 由于 $\ln x$ 在 $x_0=0$ 处无意义,因此不妨对函数 $c(x)=\ln (1-x)$ 在 $x_0=0$ 处进行展开:

\begin{align*}
c(x) &= \sum_{i=1}^{+\infty} (-\dfrac{x^i}{i}) \\
&= -(x+\dfrac{x^2}{2}+\dfrac{x^3}{3}+\cdots)
\end{align*}

\noindent 不妨计算到 $2$ 次方项:

$$c(0.1) \approx -(10^{-1}+5 \times 10^{-3})=-0.105$$

\noindent 因此 $c \approx 0.105$.

\noindent 整理得 $a \approx 0.1105$,$b \approx 0.1111$,$c \approx 0.105$.

\noindent 故 $c \lt a \lt b$,选 C.

\end{document}